\documentclass[11pt]{article}

\usepackage[a4paper, left=25mm, right=25mm, top=25mm, bottom=25mm]{geometry}
\usepackage[T1]{fontenc}
\usepackage[utf8]{inputenc}
\usepackage[czech]{babel}

\usepackage{amsmath, amssymb}
\usepackage{graphicx}
\usepackage{float}
\usepackage{setspace}

\onehalfspacing

\title{Porovnání klasického SVD a randomizovaného SVD}
\author{Aruzhan Abilmazhinova \\ ABI0008}
\date{Numerická lineární algebra 2 (NLA2) \\ Zimní semestr 2025}

\begin{document}

\maketitle

\section{Úvod}

Singulární rozklad matice (Singular Value Decomposition, SVD) patří mezi základní nástroje numerické lineární algebry a nachází široké využití v oblastech, jako je zpracování dat, redukce dimenze, komprese informací nebo řešení úloh nejmenších čtverců.

Klasické algoritmy pro výpočet SVD jsou numericky stabilní a velmi přesné, avšak jejich výpočetní a paměťová náročnost může být problematická při práci s velkými maticemi. V posledních letech se proto objevují randomizované algoritmy, které umožňují aproximovat SVD s výrazně nižší výpočetní složitostí a paměťovými nároky, a to za cenu malé ztráty přesnosti.

Cílem tohoto projektu je porovnat klasický SVD a randomizovaný SVD z hlediska:
\begin{itemize}
    \item výpočetního času,
    \item paměťové náročnosti,
    \item přesnosti aproximace.
\end{itemize}

\section{Popis použitých metod}

\subsection{Klasický SVD}

Klasický singulární rozklad matice $A \in \mathbb{R}^{m \times n}$ má tvar
\[
A = U \Sigma V^T,
\]
kde $U \in \mathbb{R}^{m \times r}$ a $V \in \mathbb{R}^{n \times r}$ jsou ortogonální matice, $\Sigma \in \mathbb{R}^{r \times r}$ je diagonální matice singulárních hodnot a $r = \min(m,n)$.

V projektu je klasický SVD realizován pomocí knihovny \texttt{SciPy}, která využívá efektivní implementace z knihovny LAPACK. Tato metoda poskytuje velmi přesné výsledky, avšak její výpočetní složitost je řádově $\mathcal{O}(mn\min(m,n))$.

\subsection{Randomizovaný SVD}

Randomizovaný SVD je aproximační metoda, která nejprve promítne původní matici do prostoru menší dimenze pomocí náhodné projekce. Následně se SVD počítá pouze na této menší matici.

Základní kroky algoritmu jsou:
\begin{enumerate}
    \item náhodná projekce původní matice,
    \item volitelné power iterace pro zvýšení přesnosti,
    \item ortogonalizace pomocí QR rozkladu,
    \item výpočet SVD menší matice,
    \item zpětná projekce do původního prostoru.
\end{enumerate}

Tento přístup výrazně snižuje výpočetní i paměťovou náročnost, zejména pokud má matice nízkou efektivní hodnost.

\section{Testovací data}

Testování bylo provedeno na třech typech matic:
\begin{itemize}
    \item náhodná hustá matice,
    \item matice s nízkou hodností (low-rank) s cílovou hodností $k = 20$,
    \item strukturovaná bloková matice.
\end{itemize}

Tento výběr umožňuje porovnat chování obou metod na různých typech vstupních dat.

\section{Kritéria hodnocení}

Porovnání metod bylo provedeno podle následujících kritérií:
\begin{itemize}
    \item výpočetní čas měřený pomocí \texttt{time.perf\_counter},
    \item paměťová náročnost měřená knihovnou \texttt{tracemalloc},
    \item přesnost aproximace vyjádřená relativní chybou ve Frobeniově normě:
    \[
    \frac{\lVert A - \tilde{A} \rVert_F}{\lVert A \rVert_F}.
    \]
\end{itemize}

\section{Výsledky}

Naměřené výsledky ukazují, že randomizovaný SVD je ve všech testovaných případech výrazně rychlejší než klasický SVD. Paměťová náročnost randomizovaného SVD je podstatně nižší, zejména u větších a strukturovaných matic.

U matic s nízkou hodností byla dosažena velmi dobrá přesnost aproximace a rozdíl oproti klasickému SVD je zanedbatelný. U náhodných hustých matic je chyba mírně vyšší, avšak stále přijatelná vzhledem k výrazné úspoře výpočetního času.

\begin{figure}[H]
\centering
\includegraphics[width=0.85\textwidth]{speed_comparison.png}
\caption{Srovnání výpočetního času klasického a randomizovaného SVD.}
\end{figure}

\section{Závěr}

V této práci bylo provedeno srovnání klasického algoritmu SVD a jeho randomizované varianty z hlediska výpočetního času, paměťové náročnosti a přesnosti aproximace. Výsledky potvrzují, že randomizované SVD představuje efektivní alternativu ke klasickému SVD, zejména pro velké a nízko-hodnostní matice, kde je kladen důraz na rychlost a paměťovou úspornost výpočtu.

\end{document}
